% -*- coding-system:utf-8 
% LATEX PREAMBLE --- needs to be imported manually
\documentclass[12pt]{article}
\special{papersize=3in,5in}
\usepackage[utf8]{inputenc}
\usepackage{amssymb,amsmath}
\pagestyle{empty}
\setlength{\parindent}{0in}

%%% commands that do not need to imported into Anki:
\usepackage{mdframed}
\newcommand*{\xfield}[1]{\begin{mdframed}\centering #1\end{mdframed}\bigskip}
\newenvironment{note}{}{}
\newcommand*{\tags}[1]{\paragraph{tags: }#1} 
% END OF THE PREAMBLE
\begin{document}

\begin{note}
	\tags{identite trigo integrales an3}
	\xfield{$\int sin^2x cosx dx$ ?}
	\xfield{$sin^3x$}
\end{note}

\begin{note}
	\tags{identite trigo an3}
	\xfield{$sin^2x$ ?}
	\xfield{$\frac{1-cos(2x}{2}$}
\end{note}

\begin{note}
	\tags{identite trigo an3}
	\xfield{$cos 2x$ ?}
	\xfield{$cos^2-sin^2$}
\end{note}

\begin{note}
	\tags{identite trigo integrales an3}
	\xfield{$\int\limits_0^{2\pi} cos^n$, n impair ?}
	\xfield{$0$}
\end{note}

\begin{note}
	\tags{identite trigo integrales an3}
	\xfield{$\int sin^3x dx$?}
	\xfield{$\frac{cos^3 x}{3} - cos x$}
\end{note}

\begin{note}
	\tags{diff an3}
	\xfield{gradient de f : $grad f(x) = \nabla f(x)$ ?}
	\xfield{$\left(\frac{\delta f}{\delta x_1}(x),...,\frac{\delta f}{\delta x_n}(x)\right) \in \mathbb{R}^n$}
\end{note}

\begin{note}
	\tags{diff an3}
	\xfield{laplacien de f : $\Delta f(x)$ ?}
	\xfield{$\sum\limits_{i = 1}^n \frac{\delta^2f}{\delta x_i^2} (x) \in \mathbb{R}$}
\end{note}

\begin{note}
	\tags{diff an3}
	\xfield{divergence de f : $div F(x) = \nabla \cdot F$ ?}
	\xfield{$\sum\limits_{i = 1}^n \frac{\delta f}{\delta x_i} (x) \in \mathbb{R}$}
\end{note}

\begin{note}
	\tags{diff an3}
	\xfield{rotationnel de f (2D) : $rot F(x) = \nabla \times F$ ?}
	\xfield{$\frac{\delta F_2}{\delta x_1}(x)-\frac{\delta F_1}{\delta x_2}(x)$}
\end{note}

\begin{note}
	\tags{diff an3}
	\xfield{rotationnel de f (3D) : $rot F(x) = \nabla \times F$ ?}
	\xfield{$\left(\frac{\delta F_3}{\delta x_2}(x)-\frac{\delta F_2}{\delta x_3}(x),
	\frac{\delta F_1}{\delta x_3}(x)-\frac{\delta F_3}{\delta x_1}(x),
	\frac{\delta F_2}{\delta x_1}(x)-\frac{\delta F_1}{\delta x_2}(x)\right)$ \\
	ou \\
	$\left( \frac{\delta}{\delta x_1}, \frac{\delta}{\delta x_2}, \frac{\delta}{\delta x_3} \right) \times \left(F_1,F_2,F_3\right)$}
\end{note}

\begin{note}
	\tags{diff an3}
	\xfield{$f \in C^2(\Omega)$  \\
	div grad $f$ ?}
	\xfield{$\nabla f$}
\end{note}

\begin{note}
	\tags{diff an3}
	\xfield{$n=3,f \in C^2(\Omega) , F \in C^2(\Omega;\mathbb{R}^3 )$ \\
	rot grad $f$ ?}
	\xfield{rot grad $f$ = 0 \\ 
	div rot $F$ = 0}
\end{note}

\begin{note}
	\tags{diff an3}
	\xfield{$f \in C^2(\Omega),g \in C^2(\Omega)$  \\
	div ($f$ grad $g$) ?}
	\xfield{$f\Delta g$ + grad $f \cdot$ grad $g$}
\end{note}

\begin{note}
	\tags{diff an3}
	\xfield{$f,g \in C^1(\Omega)$ \\
	grad $(fg)$?}
	\xfield{$f$ grad $g$ + $g$ grad $f$}
\end{note}

\begin{note}
	\tags{diff an3}
	\xfield{$f \in C^1(\Omega) , F \in C^1(\Omega;\mathbb{R}^n )$  \\
	div $(fF)$?}
	\xfield{$f$ div $F$ + $F \cdot$ grad $f$}
\end{note}

\begin{note}
	\tags{diff an3}
	\xfield{$n=3, F \in C^2(\Omega;\mathbb{R}^3 )$ \\
	rot rot $F$ ?}
	\xfield{$-\Delta F$ + grad div $F$}
\end{note}

\begin{note}
	\tags{diff an3}
	\xfield{$n=3, f \in C^1(\Omega) , F \in C^1(\Omega;\mathbb{R}^3)$ \\
	rot $(fF)$}
	\xfield{grad $f \times F + f$ rot $F$}
\end{note}

\begin{note}
	\tags{curvilignes an3}
	\xfield{Intégrale d'une fonction $f: \Gamma \to \mathbb{R}$ le long de
    $\Gamma$ (courbe)}
	\xfield{$\int\limits_\Gamma f dl = \int\limits_a^b f(\gamma(t)) ||\gamma'(t)||
    dt$\\
où $\gamma$ est une paramétrisation de $\Gamma$ et $||\gamma'(t)|| =
\sqrt[Root]{\sum\limits_{\nu = 1}^n (\gamma'_{\nu}(t)^2)}$}
\end{note}

\begin{note}
	\tags{curvilignes an3}
	\xfield{Intégrale d'un champ vectoriel $F: \Gamma \to \mathbb{R}^n$ le long de
    $\Gamma$ (courbe)}
	\xfield{$\int\limits_\Gamma F dl = \int\limits_a^b F(\gamma(t)) \cdot \gamma'(t)
    dt$\\
où $\gamma$ est une paramétrisation de $\Gamma$}
\end{note}

\begin{note}
	\tags{potentiel an3}
	\xfield{Définition de $F$ dérive d'un potentiel}
	\xfield{$F(x) = grad f(x) = \left( \frac{\delta f}{\delta x_1} (x), ... ,
      \frac{\delta f}{\delta x_n} (x) \right)$}
\end{note}

\begin{note}
	\tags{potentiel an3}
	\xfield{Condition(s) pour que $F$ dérive d'un potentiel sur $\Omega$}
	\xfield{$\text{rot} F = 0$, suffisant que si $\Omega$ est au moins simplement connexe}
\end{note}

\begin{note}
	\tags{potentiel an3}
	\xfield{Recette pour trouver si $F = (F_1,F_2,...,F_n)$ dérive d'un potentiel et si tel est le cas, trouver le dit
    potentiel}
	\xfield{\begin{itemize}
      \item[]{Regarder si le domaine de définition est bien simplement connexe}
      \item[]{Si le Rot de $F$ = 0, alors $F$ dérive d'un potentiel}
      \item[]{Mettre en égalité chaque dérivée partielle de $f(x_1,x_2,...,x_n)$ en lien avec leur
          valeur de $F$ (e.g $\frac{\delta f}{\delta x_1} = F_1$)}
        \item[]{Intégrer la première par rapport à $x_1$ (e.g. $f(x_1,x_2,...,x_n)
            = \int F_1 + \alpha(x_2,...,x_n)$), puis mettre le résultat dans la
            deuxième et procéder ainsi de suite jusqu'à ne plus avoir de
            fonction $\alpha$}
          \item[]{Si le domaine n'est pas clair, le couper en plusieurs parties
              et essayer de trouver des valeures de constantes permettant de
              faire le lien}
            \item[]{Si l'on pense que $F$ ne dérive pas d'un potentiel, essayer
                de l'intégrer sur certaines boucles, comme le cercle unité ou le
              carré unité}
    \end{itemize}}
\end{note}

\begin{note}
	\tags{Green an3}
	\xfield{Théorème de Green avec $\Omega \subset \mathbb{R}^2$ un domaine
    régulier dont le bord $\delta \Omega$ est orienté positivement.}
	\xfield{$\iint\limits_{\Omega} rot F (x_1,x_2) dx_1dx_2 =
    \int\limits_{\delta \Omega} F \cdot dl$}
\end{note}

\begin{note}
	\tags{Divergence an3}
	\xfield{Théorème de la divergence dans le plan}
	\xfield{$\iint\limits_{\Omega} \text{div} F(x_1,x_2) dx_1dx_2 =
    \int\limits_{\delta \Omega} (F \cdot \nu) dl$\\
où $\nu$ est un champ de normales extérieures unité à $\delta \Omega$}
\end{note}

\begin{note}
	\tags{surface an3}
	\xfield{Intégrale de $f$ (champ vectoriel continu) sur la surface régulière $\Sigma \subset \mathbb{R}^3$}
	\xfield{$\iint\limits_{\Sigma} f ds = \iint\limits_A f(\sigma (u,v)) ||
    \sigma_u \wedge \sigma_v|| du dv$\\
où $\sigma = \sigma(u,v): \bar{A} \to \mathbb{R}^3$ est une paramétrisation de $\Sigma$}
\end{note}

\begin{note}
	\tags{surface an3}
	\xfield{Intégrale de $F$ (champ vectoriel continu) sur la surface régulière orientable $\Sigma \subset \mathbb{R}^3$}
	\xfield{$\iint\limits_{\Sigma} F ds = \iint\limits_A \left[ F(\sigma (u,v)) \cdot 
    \sigma_u \wedge \sigma_v \right] du dv$\\
où $\sigma = \sigma(u,v): \bar{A} \to \mathbb{R}^3$ est une paramétrisation de $\Sigma$}
\end{note}

\begin{note}
	\tags{Divergence an3}
	\xfield{Théorème de la divergence en 3d}
	\xfield{$\iiint\limits_{\Omega} \text{div} F(x,y,z) dxdydz =
    \iint\limits_{\delta \Omega} (F\cdot \nu) ds$\\
où $\nu$ est la normale extérieure unité à $\Omega$}
\end{note}

\begin{note}
	\tags{Stokes an3}
	\xfield{Théorème de Stokes, sur la surface régulière $\Sigma \subset \mathbb{R}^3$}
	\xfield{$\iint\limits_{\Sigma} \text{rot} F \cdot ds =
    \int\limits_{\delta \Sigma} F\cdot dl$}
\end{note}

\begin{note}
	\tags{Fourier an3}
	\xfield{Série de Fourier}
	\xfield{$F_N f(x) =  \frac{a_0}{2} + \sum\limits_{n=1}^N \lbrace a_n \cos\left(\frac{2\pi n}{T} x \right) + b_n \sin\left(\frac{2\pi n}{T} x \right)\rbrace$\\
où\\
$a_n = \frac{2}{T} \int\limits_0^T f(x)\cos\left(
      \frac{2\pi n}{T} x \right) dx$, $n=0,1,2,...$\\
$b_n = \frac{2}{T} \int\limits_0^T f(x)\sin\left(
      \frac{2\pi n}{T} x \right) dx$, $n=1,2,...$\\
$T$ étant la période de la fonction.\\
Notons également que si $f(x)$ est paire,
$b_n = 0$ et si $f(x)$ est impaire $a_n = 0$}
\end{note}

\begin{note}
	\tags{Fourier an3}
	\xfield{Fourier en notation complexe}
	\xfield{$Ff(x) = \sum\limits_{n=-\infty}^{\infty} c_n e^{i \frac{2\pi n}{T}
      x}$\\
où $c_n = \frac{1}{T}\int\limits_0^T f(x) e^{-i \frac{2\pi n}{T} x} dx$}
\end{note}


\begin{note}
	\tags{Fourier an3}
	\xfield{Identité de Parseval}
	\xfield{$\frac{2}{T} \int\limits_0^T(f(x))^2 dx = \frac{a_0^2}{2} +
    \sum\limits_{n=1}^{\infty} (a_n^2 + b_n^2)$}
\end{note}


\end{document}
