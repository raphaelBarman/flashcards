% -*- coding-system:utf-8 
% LATEX PREAMBLE --- needs to be imported manually
\documentclass[12pt]{article}
\special{papersize=3in,5in}
\usepackage[utf8]{inputenc}
\usepackage{amssymb,amsmath}
\pagestyle{empty}
\setlength{\parindent}{0in}

%%% commands that do not need to imported into Anki:
\usepackage{mdframed}
\newcommand*{\xfield}[1]{\begin{mdframed}\centering #1\end{mdframed}\bigskip}
\newenvironment{note}{}{}
\newcommand*{\tags}[1]{\paragraph{tags: }#1} 
% END OF THE PREAMBLE
\begin{document}

\begin{note}
	\tags{identite,jacobien}
	\xfield{Quel est le déterminant de la Jacobienne du changement de cordonnées en coordonées spérique ?\\
	$$\det(J_G(r,\theta,\varphi)) = ?$$}
	\xfield{$$\det(J_G(r,\theta,\varphi)) = r^2 \cdot \sin(\theta)$$}
\end{note}
\begin{note}
	\tags{identite,jacobien}
	\xfield{Quel est le déterminant de la Jacobienne du changement de cordonnées en coordonées cylindrique ?\\
	$$\det(J_G(r,\varphi,z)) = ?$$}
	\xfield{$$\det(J_G(r,\varphi,z)) = r$$}
\end{note}
\begin{note}
	\xfield{Donner la forme d'une équation homogène du deuxième ordre sans coefficient constants et expliquer comment la résoudre.}
	\xfield{Les équations homogène de la forme :
$$y''+p(x)y'+q(x)y=0$$
avec p,q des fonction continues sur un intervalle $I$, peuvent être résolues, si l'on connaît une solution non nulle $y_1$, en posant
$$y(x) = U(x)\cdot y_1(x)$$
avec $U$ une primitive d'une nouvelle fonction inconnue $u$.\\
En développant,on obtient finalement :
\begin{align*}
y_1(x) \cdot u'(x) + (2\cdot y_1'(x) + p(x)\cdot y_1(x))\cdot u(x) &= 0
\end{align*}
que l'on sait résoudre.}
\end{note}
\begin{note}
	\xfield{Donner la forme d'une équation de Ricatti et expliquer comment la résoudre.}
	\xfield{C'est une équation de la forme :
$$y'=a(x)\cdot y^2 + b(x) \cdot y+c(x)$$
avec $a,b,c$ des fonctions continues sur un intervalle $I$.
On ne peut pas la résoudre en général, sauf si l'on connaît déjà une solution.\\
Donné une solution $y_1(x)$, on pose :
$$y(x)=y_1(x)+\frac{1}{u(x)}$$
avec $u(x)$ inconnu.\\
En développant, on obtient finalement :
\begin{align*}
&u'+(2\cdot a \cdot y_1 + b)\cdot u = -a
\end{align*}que l'on sait résoudre.}
\end{note}
\begin{note}
\xfield{$$\nabla \left( \frac{g}{h} \right) =  ?$$}
\xfield{$$\nabla \left( \frac{g}{h} \right) = \frac{1}{h}\nabla g + \frac{g}{h^2} \nabla h$$ }
\end{note}
\begin{note}
\xfield{Donner le théorème des fonctions implicites}
\xfield{Une équation $F(x_0,y_0) = 0$ avec $\frac{\delta F}{\delta y} \neq 0$ défini une fonction $y=f(x)$ implicite au voisinage de $(x_0,y_0)$. Et on a :
$$F(x, f(x)) = 0 \Rightarrow f(x_0) = y_0$$
Et aussi/surtout :
$$f'(x) = - \frac{\delta_xF (x,f(x))}{\delta_yF(x,f(x))}$$}
\end{note}
\begin{note}
\xfield{Donner la définition de la dérivée directionnelle.}
\xfield{La dérivée directionnelle en $(x_0,y_0)$, dans la direction du vecteur $\mathbf{e}$ s'obtient soit en appliquant la définition de la dérivée. 
$$\delta_e f(x,y) = \lim_{t \to 0} \frac{f(x_0+t\mathbf{e}_x,y_0 + t\mathbf{e}_y) - f(x_0,y_0)}{t}$$
Ou en utilisant que
$$\delta_e f(x_0,y_0) = \nabla f(x_0,y_0) \cdot \mathbf{e}$$ }
\end{note}

\begin{note}
\xfield{Quel $\mathbf{e}$ choisir pour obtenir la dérivée directionnelle maximale (minimale) ?}
\xfield{La dérivée directionnelle est maximale (minimale) si elle point dans le sens (sens opposé) de $\nabla f(x_0,y_0)$, et donc :
\begin{align*}
mathbf{e} = \frac{\nabla f(x_0,y_0)}{\vert \vert \nabla f(x_0,y_0) \vert \vert} \tag*{(-$\mathbf{e}$ pour minimal)}
\end{align*}La pente vaut $\vert \vert \nabla f(x_0,y_0) \vert \vert$ en ce point. }
\end{note}
\begin{note}
\xfield{Expliquer comment dérivier $$F(t) = \int_{a(t)}^{b(t)} f(x,t)dx $$}
\xfield{$$F(t) = \int_{a(t)}^{b(t)} f(x,t)dx $$
$$\frac{d}{dt} F(t) = -f(a(t),t)a'(t) + f(b(t),t)b'(t) + \int_{a(t)}^{b(t)} \delta _t f(x,t) dx$$
Dans les cas particulier ou $a(t)$ et/ou $b(t)$ sont constant les termes avec leur dérivées sautent. Particulièrement si $a(t) = a, b(t) = b$ :
$$\frac{d}{dt} F(t) = \int_a^b \delta_t f(x,t)dx$$}
\end{note}

\begin{note}
\xfield{Points critiques : donner les valeurs des divers $\Lambda$ et leur signification pour $n=2$ et $n=3$}
\xfield{\underline{Pour n=2 :}\\
On a :
\begin{align*}
\Lambda_1 &= \delta x^2 f(x,y)  = H_{11}(f)\\
\Lambda_2 &= det(H_f(x_0,y_0)) 
\end{align*}
\begin{center}
\begin{tabular}{c|c|c}
$\Lambda_2$ & $\Lambda_1$ & Conclusion \\ 
\hline
$>$ 0 & $>$ 0 & minimum local \\ 
$>$ 0 & $<$ 0 & maximum local \\ 
$<$ 0 & ? & point-selle \\ 
= 0 & ? & ? \\ 
\end{tabular} 
\end{center}
\underline{Pour n=3 :}\\
On a :
\begin{align*}
\Lambda_1 &= \dfrac{\partial^2 f}{\partial x^2} f(x_0,y_0,z_0)  = H_{11}(f)\\
\Lambda_2 &= \det\begin{pmatrix}
  \dfrac{\partial^2 f}{\partial x^2} & \dfrac{\partial^2 f}{\partial x\,\partial y}\\
  \dfrac{\partial^2 f}{\partial y\,\partial x} & \dfrac{\partial^2 f}{\partial y^2} \\[2.2ex]
\end{pmatrix} = \dfrac{\partial^2 f}{\partial x^2} \cdot \dfrac{\partial^2 f}{\partial y^2} - \dfrac{\partial^2 f}{\partial x\,\partial y} \cdot \dfrac{\partial^2 f}{\partial y\,\partial x}\\
\Lambda_3 &= \det(H_f(x_0,y_0,z_0)) 
\end{align*}
\begin{center}
\begin{tabular}{c|c|c|c}
$\Lambda_3$ & $\Lambda_2$ & $\Lambda_1$ & Conclusion \\ 
\hline
$>$ 0& $>$ 0 & $>$ 0 & minimum local \\ 
$<$ 0 & $>$ 0 & $<$ 0 & maximum local \\ 
$\neq 0$ & ? & ? & point-selle \\ 
\end{tabular} 
\end{center}}
\end{note}

\begin{note}
\xfield{Dire où peuvent se trouver les extremums absolus pour une fonction continue sur un domain borné et fermé.}
\xfield{Ils peuvent se trouver :
\begin{itemize}
\item Sur un extremum local
\item Sur un point ou f n'est pas différentiable
\item Sur le bord du domaine 
\end{itemize}}
\end{note}

\begin{note}
\xfield{Expliquer comment trouver un extremum sous contrainte d'une autre fonction.}
\xfield{Si $f(x,y)$ est une fonction à maximiser sous la contrainte $g(x,y) = 0$ on peut définir la fonction de Lagrange :
$$F(x,y,\lambda) = f(x,y) - \lambda g(x,y)$$
Ou :
$$F(x,y,\lambda, \mu) = f(x,y) - \lambda g(x,y) - \mu h(x,y)$$
Si il y a plusieurs contraintes.
On pose ensuite $\nabla F(x,y,\lambda, (\mu)) = \vec{0}$: Et on résout.}
\end{note}

\begin{note}
\xfield{Comment trouver la fonction d'un plan tangent ?}
\xfield{Donnée une fonction de la forme $z=f(x,y)$ (il faut isoler z si ce n'est pas le cas), le plan tangent en $x_0,y_0,z_0$ (également donné) On sait que
$$ z = f(x_0,y_0) + \frac{\delta f}{\delta x}f(x_0,y_0)(x-x_0) + \frac{\delta f}{\delta y}f(x_0,y_0)(y-y_0) $$ forme un plan tangent.}
\end{note}

\begin{note}
\xfield{Comment trouver la fonction d'un plan tangent d'une fonction implicite ?}
\xfield{Si on a une surface définie par une équation du type $F(x,y,z) = 0$ et que $\frac{dF}{dz}(x_0,y_0,z_0) \neq 0$ on peut définir l'équation du plan tangent comme :
$$\nabla F(x_0,y_0,z_0) \cdot (x-x_0,y-y_0,z-z_0) = 0$$
Ou $\cdot$ est le produit scalaire. Cela correspond à prendre tout vecteur \textbf{orthogonal} à la normale en ce point de la surface. Cela défini bien un plan.
}
\end{note}
\end{document}
