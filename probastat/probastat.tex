% -*- coding-system:utf-8 
% LATEX PREAMBLE --- needs to be imported manually
\documentclass[12pt]{article}
\special{papersize=3in,5in}
\usepackage[utf8]{inputenc}
\usepackage{amssymb,amsmath}
\pagestyle{empty}
\setlength{\parindent}{0in}

%%% commands that do not need to imported into Anki:
\usepackage{mdframed}
\newcommand*{\xfield}[1]{\begin{mdframed}\centering #1\end{mdframed}\bigskip}
\newenvironment{note}{}{}
\newcommand*{\tags}[1]{\paragraph{tags: }#1} 
% END OF THE PREAMBLE
\begin{document}
\begin{note}
	\tags{combinatorics}
	\xfield{Permutation with repetition (order matters)} %% Ajouter le devant de la carte entre {}
	\xfield{If we have $n$ distinct object, the number of permutation (with
    repetition) of length $r \le n$ is $$r^n$$} %% Ajouter le dos de la carte entre {}
\end{note}

\begin{note}
	\tags{combinatorics}
	\xfield{Permutation without repetition (order matters)} %% Ajouter le devant de la carte entre {}
	\xfield{Number of permutation of $k$ objects amongst $n$:
$$\frac{n!}{(n-k)!}$$} %% Ajouter le dos de la carte entre {}
\end{note}

\begin{note}
	\tags{combinatorics}
	\xfield{Combination without repetition (order doesn't matter)} %% Ajouter le devant de la carte entre {}
	\xfield{Number of combination with $k$ objects chosen amongst $n$
$$\binom{n}{k} = \frac{n!}{(n-k)!k!}$$} %% Ajouter le dos de la carte entre {}
\end{note}

\begin{note}
	\tags{combinatorics}
	\xfield{Combination with repetition} %% Ajouter le devant de la carte entre {}
	\xfield{Number of combination of $n$ undistinguishable balls in $k$
    distinguishable urns:
$$C^{n}_{n+k-1} = \frac{(n+k-1)!}{n!(k-1)!}$$} %% Ajouter le dos de la carte entre {}
\end{note}

\begin{note}
	\tags{random variable, discrete, definition}
	\xfield{probability density function (PDF)} %% Ajouter le devant de la carte entre {}
	\xfield{$f_X(x)=P(X=x)$} %% Ajouter le dos de la carte entre {}
\end{note}

\begin{note}
	\tags{random variable, definition}
	\xfield{cumulative distribution function (CDF)} %% Ajouter le devant de la carte entre {}
	\xfield{$F_X(t)=P(X \le t)$} %% Ajouter le dos de la carte entre {}
\end{note}

\begin{note}
	\tags{random variable, definition}
	\xfield{Quantile} %% Ajouter le devant de la carte entre {}
	\xfield{For a probability $0<p<1$, the $p$ quantile of $F_X$ is 
$$x_p = \text{inf}\lbrace x: F_X(x) \ge p \rbrace$$
very often $F_X(x_p) = p$} %% Ajouter le dos de la carte entre {}
\end{note}

\begin{note}
	\tags{random variable, continuous,definition}
	\xfield{Continuous random variable (PDF and CDF)} %% Ajouter le devant de la carte entre {}
	\xfield{$$F_X(x) = \int\limits_{-\infty}^x f_X(t) dt$$
so
$$f_X(x) = \frac{dF_X(x)}{dx}$$} %% Ajouter le dos de la carte entre {}
\end{note}

\begin{note}
	\tags{random variable, continuous,joint,definition}
	\xfield{Joint random variables, $F_{X,y} = ?$ and $f_{X,Y} = ?$} %% Ajouter le devant de la carte entre {}
	\xfield{$$F_{X,Y}(x,y) = P(X \le x, Y \le y) = \int\limits_{-\infty}^y
    \int\limits_{-\infty}^x f_{X,Y}(s,t)ds dt$$
    and
$$f_{X,Y}(x,y)=\frac{\delta^2}{\delta x \delta y}F_{X,Y}(x,y)$$} %% Ajouter le dos de la carte entre {}
\end{note}

\begin{note}
	\tags{random variable, continuous,discrete,joint,definition}
	\xfield{Marginal probability density function and condtional density function} %% Ajouter le devant de la carte entre {}
	\xfield{$$f_X(x) = \begin{cases} \sum\limits_y f_{X,Y}(x,y),\text{ discrete
        case}\\
      \int\limits_{-\infty}^{\infty}f_{X,Y}dy,\text{ continuous case}
\end{cases}$$
$$f_{Y|X}(y|x) = \frac{f_{X,Y}(x,y)}{f_X(x)},\ y \in \mathbb{R}$$} %% Ajouter le dos de la carte entre {}
\end{note}

\begin{note}
	\tags{random variable, continuous}
	\xfield{Change of variable (one dimension):\\
If $X$ is a continuous random variable of density $f_X(x)$ and if
    $Y=g(X)$ for a function $g$ real and invertible, we have...} %% Ajouter le devant de la carte entre {}
	\xfield{$$F_Y(t) = \int\limits_{g^{-1}(-\infty,y)}f_X(x)dx$$
If $g(x)$ is a strictly monotonous and continuously derivable, then the random
variable $Y=g(X)$ is a continuous random variable with density
$$f_Y(y) = \begin{cases}
f_X(g^{-1}(y))|\frac{d g^{-1}(y)}{dy}|, &\text{ if $y=g(x)$ for any $x$}\\
0,&\text{ if $y \neq g(x)$ for all $x$}
\end{cases}$$
    } %% Ajouter le dos de la carte entre {}
\end{note}

\begin{note}
	\tags{random variable, continuous, joint}
	\xfield{Change of variable (two dimension):\\
If $(X_1,X_2)$ are two jointly continuous random variable of density $f_{X_1,X_2}(x_1,x_2)$ and if
    $(Y_1,Y_2)=(g_1(X_1,X_2),g_2(X_1,X_2))$ for a function $g=(g_1,g_2)$ invertible, we have...} %% Ajouter le devant de la carte entre {}
	\xfield{$$F_{Y_1,Y_2}(y_1,y_2) = \iint\limits_{\substack{g_1(x_1,x_2)\le y_1
        \\ g_2(x_1,x_2) \le y_2}} f_{X_1,X_2}(x_1,x_2)dx_1 dx_2$$
If $g(x)$ is a continuously derivable bijection and its inverse denoted by
$h=(h_1,h_2)$, then the random couple $(Y_1,Y_2)$ is jointly continuous with
density:
$$f_{Y_1,Y_2}(y_1,y_2) = f_{X_1,X_2}(h_1(y_1,y_2),h_2(y_1,y_2)) | J(y_1,y_2) |$$
where $|J(y_1,y_2)|$ is the Jacobian of $h$ and
$$J(y_1,y_2) = \frac{\delta h_1}{\delta y_1}\frac{\delta h_2}{\delta y_2} -
\frac{\delta h_1}{\delta y_2} \frac{\delta h_2}{\delta y_1}$$
} %% Ajouter le dos de la carte entre
\end{note}

\begin{note}
\tags{distribution, discrete}
\xfield{Binomial distribution}
\xfield{$$P(X = x) = \binom{n}{x} p^x (1-p)^{n-x}$$
$E[X] = np$\\
$var[X] = np(1-p)$\\
$\mathcal{M}_X(t) = (pe^t+1-p)^n$\\
Use when $n$ (fixed) trials.
}
\end{note}

\begin{note}
\tags{distribution, discrete}
\xfield{Negative binomial distribution}
\xfield{$$P(X = x) = \binom{n-1}{x-1}p^n(1-p)^{x-n}$$
$E[X] = \frac{n}{p}$\\
$var[X] = n \cdot \frac{1-p}{p^2}$\\
$\mathcal{M}_X(t) = \left(  \frac{pe^t}{1-(1-p)e^t}\right)^n$ if $(1-p)e^t < 1$\\
Use to mesure time before last of several sucesses}
\end{note}

\begin{note}
\tags{distribution, discrete}
\xfield{Poisson distribution}
\xfield{$$P(X = x) = e^{-\lambda}\frac{\lambda^k}{k!}$$
$E[X] = \lambda$\\
$var[X] = \lambda$\\
$\mathcal{M}_X(t) = e^{\lambda(e^t-1)}$\\
Use to approximate Binomial when $n\simeq \infty$ or $n >> np$}
\end{note}

\begin{note}
\tags{distribution, continuous}
\xfield{Uniform distribution}
\xfield{$$P(X = x) = \frac{1}{b-a}\text{ if $x \in [a,b]$, $0$ else}$$
$E[X] = \frac{a+b}{2}$\\
$var[X] = \frac{(b-a)^2}{12}$\\
$\mathcal{M}_X(t) = \frac{e^{tb}-e^{ta}}{t(b-a)}$ $\forall t \neq 0$}
\end{note}

\begin{note}
\tags{distribution}
\xfield{Gamma function (not a distribution, but useful), $\Gamma(\alpha)$ = ?}
\xfield{$\Gamma(\alpha) = \int\limits_0^{\infty} u^{\alpha -1}e^{-u}du$ $\alpha >0$}
\end{note}

\begin{note}
	\tags{proba, basics}
	\xfield{Probabilité $P(A | B)$ ?}
	\xfield{$P(A | B) = \frac{P(A \cap B)}{P(B)} = \frac{P(B | A)P(A)}{P(B)}$}
\end{note}

\begin{note}
	\tags{proba,basics}
	\xfield{Formule de probabilité totale ?}
	\xfield{$P(A) = P(A | B_1)P(B_1)+...+P(A | B_n)P(B_n)$}
\end{note}

\begin{note}
	\tags{proba,basics}
	\xfield{Formule de baise ?}
	\xfield{$P(B_i | A) = \frac{P(A | B_i)P(B_i)}{P(A | B_1)P(B_1) + ... + P(A | B_n)P(B_n)}$}
\end{note}

\begin{note}
	\tags{proba,expectation}
	\xfield{Pour une Variable aléatoire continue $X$ on a $E[X] =$ ?}
	\xfield{$\int xdF_X(x)$, définie quand $E[|X|] \leq \infty $}
\end{note}

\begin{note}
	\tags{proba,expectation}
	\xfield{$E(g(X))$ pour un $X$ discret?}
	\xfield{$\sum \limits_{k \in K} g(x_k)P(X = x_k)$}
\end{note}

\begin{note}
	\tags{proba,expectation}
	\xfield{$E(g(X))$ pour un $X$ continu?}
	\xfield{$\int \limits_{-\infty}^{\infty} g(x)f_X(x)dx$}
\end{note}

\begin{note}
	\tags{proba,expectation}
	\xfield{$E(g(X,Y)$ pour des $X,Y$ continus?}
	\xfield{$\int \limits_{-\infty}^{\infty} \int \limits_{-\infty}^{\infty} g(x,y)f_{X,Y}(x,y)dxdy$}
\end{note}

\begin{note}
	\tags{proba,expectation}
	\xfield{$var[X]$ ?}
	\xfield{$E[(X-E[X])^2] = E[X^2]-E[X]^2 = cov[X,X]$}
\end{note}

\begin{note}
	\tags{proba,expectation}
	\xfield{$cov[X,Y]$ ?}
	\xfield{$E[(X-E[X])(Y-E[Y])] = E[XY] - E[X]E[Y]$}
\end{note}

\begin{note}
	\tags{proba,expectation}
	\xfield{$corr[X,Y]$ ?}
	\xfield{$\frac{cov[X,Y]}{
	\sqrt{var[X]}
	\sqrt{var[Y]}}$}
\end{note}


\begin{note}
	\tags{proba,expectation}
	\xfield{$cov[a + bX_1+ cX_2,Y]$ (linéarité de la covariance) ?}
	\xfield{$b cov[X_1,Y] + c cov[X_2,Y]$}
\end{note}

\begin{note}
	\tags{proba,expectation}
	\xfield{$var[X+Y]$ (variance et covariance)?}
	\xfield{$var[X] + var[Y]  + 2cov[X,Y]$}
\end{note}

\begin{note}
	\tags{proba,expectation}
	\xfield{$var[aX]$ (homogénéité de la variance)?}
	\xfield{$a^2 var[X]$}
\end{note}

\begin{note}
	\tags{proba,moments}
	\xfield{Soit $X$ une variable aléatoire et $t \in R$. On appelle
fonction génératrice des moments (FGM ou MGF) (aussi transformée de Laplace) la fonction définie
par ?}
	\xfield{$ \mathcal{M}_X (t) \equiv \mathcal{L}_X(t) = E[e^{tX}] $}
\end{note}

\begin{note}
	\tags{proba,moments}
	\xfield{Pour tout entier positif $k$, le moment d’ordre $k$ est ?}
	\xfield{$E[X^k]$}
\end{note}


\begin{note}
	\tags{proba,moments}
	\xfield{Pour tout entier positif $k$, le moment \textbf{\emph{centré}} d’ordre $k$ est ?}
	\xfield{$E[(X-E[X])^k]$, en particulier, si $k = 2$, on a $var[X]$}
\end{note}

\begin{note}
	\tags{proba,moments}
	\xfield{Soit X une variable aléatoire et $t \in R$. On
appelle fonction génératrice des cumulants (FGC ou CGF) la fonction définie par ?}
	\xfield{$\mathcal{K}_X(t) = \log{\mathcal{M}_X(t)} = \log{E[e^{X}]} = \sum \limits_{k=1}^{\infty} \mathit{k}_k1 \frac{t^k}{k!}$, où $k_k$ est le cumulant d'ordre $k$, on a
	$k_k = d^k \mathcal{K}_X(t)/dt^k |_{t = 0}$, et en particulier $E[X] = k_1$ et $var[X] = k_2$}
\end{note}

\begin{note}
	\tags{proba,moments}
	\xfield{Variance and expectancy in term of Moment $M_X$}
	\xfield{$$E(X) = M_X'(0)$$
$$\text{var}(X) = M_X''(0)-M_X'(0)^2$$}
\end{note}

\begin{note}
	\tags{proba,expectancy,variance}
	\xfield{If $X_1,\ \hdots, X_n$ are independent and all have mean $\mu$ and
    variance $\sigma^2$ , then $E(\bar{X}) = ?$ and $\text{var}(\bar{X})= ?$}
	\xfield{$$E(\bar{X}) = \mu$$
$$\text{var}(\bar{X}) = \frac{\sigma^2}{n}$$}
\end{note}

\begin{note}
  \tags{random variable}
\xfield{What are the CDF of $X_{(n)}=\text{max}(X_1,...,X_n)$ and $X_{(0)}=\text{min}(X_1,...,X_n)$}
\xfield{$$F_{X_{(n)}}(x)=F_{X_1}(x)...F_{X_n}(x)$$
$$F_{X_{(0)}}(x)=1-(1-F_{X_1}(x))...(1-F_{X_n}(x))$$}
\end{note}

\begin{note}
  \tags{random variable, theorem, normal}
  \xfield{Central limit theorem for $X_1,...X_n$ of expectancy $\mu$ and variance $\sigma^2$}
  \xfield{$$Z_n= \frac{X_1+...+X_n-n\mu}{\sigma \sqrt{n}} =
    \frac{\bar{X}-\mu}{\sigma/\sqrt{n}}$$
And $$\lim\limits_{n\to \infty} P(Z_n \le z) = \varPhi(z)$$}
\end{note}

\begin{note}
  \tags{random variable, normal}
\xfield{Given a Normal distribution $\mathcal{N}(\mu,\sigma^2)$, how to find
  $P(X \le x)$ using the standard normal distribution ($\mathcal{N}(0,1)$) ?}
\xfield{$P(X \le x) = \varPhi(\frac{x-\mu}{\sigma})$}
\end{note}

\end{document}
